\documentclass[10pt]{article}

\usepackage[utf8x]{inputenc}
\usepackage{hyperref}

\title{Proiect pentru lucrarea de licență: interfață web pentru {\tt
    RMT} și alte instrumente similare ca mod de funcționare}

\author{Ștefan Ciobâcă}

\begin{document}

\maketitle

{\tt RMT} este un instrument în linia de comandă scris în C++ care
poate demonstra proprietăți de accesibilitate în sisteme de termeni cu
constrângeri. Programul {\tt RMT} primește la intrarea standard date
de intrare în format RMT și produce la ieșirea standard un text care
reprezintă rezultatul execuției. Scopul acestui proiect este de a crea
o interfață web pentru {\tt RMT}, care să poate fi ușor accesibilă.

Descriere proiect: o interfață web cu două textbox-uri, unul pentru
datele de intrare oferite RMT și altul pentru datele de ieșire produse
de RMT. La apăsarea unui buton, datele de intrare sunt trimise către
{\tt RMT}, acesta le procesează și datele de ieșire sunt afișate în
textbox-ul de ieșire.

Criterii de succes:

\begin{enumerate}

\item interfața web este stabilă;

\item textbox-ul cu date de intrare oferă syntax highlighting
  specific {\tt RMT}, paranthesis matching;

\item datele de ieșire sunt afișate pe măsură ce sunt produse de
  {\tt RMT} (execuția poate fi lungă, și din acest motiv datele de
  ieșire trebuie afișate pe măsură ce sunt produse, nu toate la
  sfârșit);

\item permite selectarea opțiunilor pentru linia de comandă aferente
  rulării {\tt RMT};

\item permite întreruperea execuției unei comenzi;

\item sistem flexibil care să permită ușor folosirea altor tool-uri
  care funcționează după același principiu (se va exemplifica prin
  folosirea sistemului pentru înca două tool-uri la alegere);

\item sistem pentru predefinirea unor exemple pentru date de intrare
  care pot fi încarcate rapid în textbox-ul de intrare.

\end{enumerate}

Un exemplu de interfață similară ca funcționalitate este inferfața web
pentru Z3: \url{http://rise4fun.com/Z3/}.

Sistemul {\tt RMT} este descris la adresa:
\url{http://profs.info.uaic.ro/~stefan.ciobaca/rmt/}.

\end{document}
