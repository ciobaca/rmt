\documentclass[10pt,a5paper]{article}
\usepackage[utf8x]{inputenc}
\usepackage[math]{iwona}
\usepackage[T1]{fontenc}

\usepackage[top=0.5in, bottom=0.75in, left=0.5in, right=0.5in]{geometry}

\usepackage{mathpartir}

\usepackage[inline]{enumitem}
\usepackage{hyperref}

\usepackage{amssymb}
\usepackage{amsmath}

\usepackage{stmaryrd}
\usepackage{xspace}

\usepackage{thmtools}
\usepackage{thm-restate}

\usepackage[obeyDraft]{todonotes}
\usepackage{bussproofs}

\newcommand{\R}{\mathcal{R}}
\newcommand{\U}{\mathcal{U}}
\renewcommand{\P}{\mathcal{P}}

\newcommand{\CT}{{\rm CT}}
\newcommand{\CF}{{\rm CF}}

\newcommand{\eqbydef}{\triangleq}
\newcommand{\tsem}[1]{[\![{#1}]\!]}
\newcommand{\sem}[1]{\lfloor\!\!\lfloor{#1}\rfloor\!\!\rfloor}
\newcommand{\gtran}[1]{\leadsto_{#1}}
\newcommand{\gpath}[1]{\circ_{#1}}
\newcommand{\rform}[2]{{#1}\Rightarrow{#2}}
\newcommand{\rrule}[3]{{#1}\twoheadrightarrow{#2}{\tt~if~}#3}
\newcommand{\crule}[3]{\ensuremath{{#1}\Mapsto{#2}\texttt{~if~}{#3}}}

\newcommand{\lbrbrak}{\left<}
\newcommand{\rbrbrak}{\right>}

\newcommand{\ct}[2]{\ensuremath{\lbrbrak{#1}\,\middle|\,{#2}\rbrbrak}}
\newcommand{\pair}[2]{\langle{#1},{#2}\rangle}
\newcommand{\var}{\mathit{var}}
\newcommand{\Sen}{\mathit{Sen}}
\newcommand{\bltn}[1]{{#1}^{\sf b}}
\newcommand{\cons}[1]{{#1}^{\sf c}}
\newcommand{\der}[1]{{#1}^{\rm der}}
\newcommand{\hd}{{\it hd}}
\newcommand{\ls}{{\it ls}}
\newcommand{\dom}{{\it dom}}

\newcommand{\Bool}{{\it Bool}\xspace}
\newcommand{\Int}{{\it Int}\xspace}
\newcommand{\Cfg}{{\it Cfg}\xspace}

\newcommand{\INT}{{\it INT}\xspace}

\newcommand{\limplies}{\rightarrow}
\renewcommand{\iff}{\leftrightarrow}
\newcommand{\True}{\top}
\newcommand{\False}{\bot}

\newcommand{\nf}{\mathord{\Downarrow}}

\newcommand{\demEntails}{\vdash^{\forall}}
\newcommand{\angEntails}{\vdash^{\exists}}

\newcommand{\demModels}{\vDash^{\forall}}
\newcommand{\angModels}{\vDash^{\exists}}
\renewcommand{\models}{\vDash}

\newcommand{\semrule}[1]{\llbracket {\sf {#1}} \rrbracket}

\newcommand{\dl}[2][]{\todo[author=Dorel,color=red!10!white,#1]{#2}}
\newcommand{\dlin}[2][]{\todo[inline, author=Dorel,color=red!10!white, caption={2do}, #1]{\newline
\begin{minipage}{\textwidth-4pt}#2\end{minipage}}}
\newcommand{\stc}[2][]{\todo[author=Stefan,color=green!10!white,#1]{#2}}

\newcommand{\SHSH}{\ensuremath{\textit{SH}}}

\newcommand{\qeq}{=}

\newcommand{\X}{\mathcal{X}}

\begin{document}

Noi avem nevoie ca $\Delta_\R$ sa satisfaca Teorema 1:

\begin{restatable}{theorem}{deltacommutes}
\label{th:der}
%
Let $\varphi \eqbydef \ct{t}{\phi}$ be a constrained term, $\R$ a
LCTRS, and $(M^\Sigma,\gtran{\R})$ the transition system defined by
$\R$.  Then
$\tsem{\Delta_\R(\varphi)} = \{\gamma' \mid \gamma \gtran{} \gamma'
\textrm{~for~some~}\gamma \in \tsem{\varphi}\}$.
\end{restatable}

Prin $\R$ inteleg sistemul de rescriere care defineste limbajul.

Sistemul $\R$ din exemplul nostru nu satisface Teorema 1. Iata un
contraexemplu. Avem ca:

\[\Delta_{\R}(\ct{{\tt~fc~(reduce~N~J~(fplus~0~0))}}{\True}) =
  \emptyset\] conform definitie noastre.

Totusi, daca $N = J$, atunci
${\tt~(fc~(reduce~N~J~(fplus~0~0)))} =_E {\tt~(fc~(fplus~0~0))}$ si,
din moment ce ${\tt~(fc~(fplus~0~0))}~\gtran{\R}~{\tt~(fc~0)}$, ar
trebui ca
\[\tsem{\ct{{\tt (fc~0)}}{N = J}} \subseteq \tsem{\Delta_\R(\ct{{\tt~fc
        ~(reduce~N~J~(fplus~0~0))}}{\True})}.\]

Chiar si in cazul in care $N >> J$, apare o regula de rescriere care
se poate aplica. De exemplu, daca $N = 10$, apare patternul
${\tt~(fc~(fplus~10~(fplus~9~\ldots))}$.

Dar, urmarind cu atentie circularitatile noastre:

\begin{verbatim}
   (pair
    (fc (f N))
    (fc (fa 0 0 N))) /\ (mle 0 N)
,
   (pair
    (reduce (mplus N 1) (mplus I 1) (fc (f I)))          // N + ... + (I+1) + (fc (f I))
    (fc (fa 0 0 N))) /\ (band (mle I N) (mle 0 I))
,
    (pair
     (reduce (mplus N 1) I (fc X))                       // N + ... + I + (fc X)
     (fc (fa X I N)))
\end{verbatim}

observam ca pe parcursul calcului nu ajunge niciun moment la un astfel
de caz, cu {\tt fc} "deasupra" lui {\tt reduce}.

Cu alte cuvinte, toti termenii care sunt intalniti pe parcursul
calculului sunt ``buni'':

\begin{definition}[Good constrained term]
  
  Let $\varphi \eqbydef \ct{t}{\phi}$ be a constrained term. We say
  that $\varphi$ is $k$-good if
  $\tsem{\Delta_\R(\varphi) \cup \Delta_R(\Delta_{E}(\varphi))} =
  \{\gamma' \mid \gamma \gtran{} \gamma' \textrm{~for~some~}\gamma \in
  \tsem{\varphi}\}$.

\end{definition}

\begin{conjecture}[Conjectura 1]

  Toti termenii obtinuti prin rescrieri simbolice din cele trei
  circularitati de mai sus sunt buni.
  
\end{conjecture}

Cred ca ar trebui sa gasim o metoda simpla si suficienta pentru a
stabili ca toti termenii intalniti pe parcursul demonstratiei sunt
$1$-good.

\begin{verbatim}
sorts State, Exp, Nexp;

subsort Int < Exp;
subsort Nexp < Exp;

signature
  fc : Exp -> Exp,
  pair : Exp Exp -> State,
  fplus : Exp Exp -> Nexp,
  fminus : Exp Exp -> Nexp,
  f : Exp -> Nexp,
  fa : Exp Exp Exp -> Nexp;

variables N : Int, I : Int, J : Int, K : Int, S : Int, A : Int, X : Int,

E : Exp, E1 : Exp, E2 : Exp, E3 : Exp,

NE : Nexp, NE1 : Nexp, NE2 : Nexp, NE3 : Nexp;

define reduce : Int Int Exp -> Exp by
  (reduce N I E) /\ (mle N I) => E,
  (reduce N I E) /\ (bnot (mle N I)) => (reduce N (mplus I 1) (fplus I E));

constrained-rewrite-system flanguage
  (fc (fplus NE1 E2)) => (fplus (fc NE1) E2),
  (fplus (fc I) E2) => (fc (fplus I E2)),
  (fc (fplus I NE2)) => (fplus I (fc NE2)),
  (fplus I (fc J)) => (fc (fplus I J)),
  (fc (fplus I J)) => (fc (mplus I J)),

  (fc (fminus NE1 E2)) => (fminus (fc NE1) E2),
  (fminus (fc I) E2) => (fc (fminus I E2)),
  (fc (fminus I NE2)) => (fminus I (fc NE2)),
  (fminus I (fc J)) => (fc (fminus I J)),
  (fc (fminus I J)) => (fc (mminus I J)),

  (fc (f NE)) => (f (fc NE)),
  (f (fc I)) => (fc (f I)),
  (fc (f N)) /\ (mle N 0) => (fc 0),
  (fc (f N)) /\ (bnot (mle N 0)) => (fc (fplus N (f (fminus N 1)))),

  (fc (fa NE1 E2 E3)) => (fa (fc NE1) E2 E3),
  (fa (fc I) E2 E3) => (fc (fa I E2 E3)),
  (fc (fa I NE2 E3)) => (fa I (fc NE2) E3),
  (fa I (fc J) E3) => (fc (fa I J E3)),
  (fc (fa I J NE3)) => (fa I J (fc NE3)),
  (fa I J (fc K)) => (fa I J K),

  (fc (fa A I N)) /\ (bnot (mle I N)) => (fc A),
  (fc (fa A I N)) /\ (mle I N) => (fc (fa (fplus A I) (fplus I 1) N));

// run in flanguage : (fc (fa 0 0 (mplus 15 15)));

show-equivalent [ 900, 1 ] in flanguage and flanguage :
   (pair
    (fc (f N))
    (fc (fa 0 0 N))) /\ (mle 0 N)
// /\       (band (mle 0 N) (mle N 2))
,
   (pair
    (reduce (mplus N 1) (mplus I 1) (fc (f I)))          // N + ... + (I+1) + (fc (f I))
    (fc (fa 0 0 N))) /\ (band (mle I N) (mle 0 I))
// /\ (band
//          (band (mle 0 N) (mle N 2))
//          (band (mle I N) (mle 0 I)))
,
    (pair
     (reduce (mplus N 1) I (fc X))                       // N + ... + I + (fc X)
     (fc (fa X I N)))
// /\ (band
//          (band (mle 0 N) (mle N 5))
//          (band (mle I N) (mle 0 I)))
with-base
     (pair (fc S) (fc S));
\end{verbatim}

   \end{document}
   
